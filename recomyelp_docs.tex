\documentclass[11pt,a4paper]{article}
\usepackage[nochapters]{classicthesis}
\usepackage[left=1in,right=1in,top=1in,bottom=1in]{geometry}

\usepackage{multicol}
\setlength{\columnsep}{.5cm}

\newcommand{\code}{\texttt}

\title{A recommender system for the Yelp dataset\\
{\small course of Algorithms for Massive Datasets}}
\author{Alessandro Clerici Lorenzini}
\date{Academic Year 2022/23}

\begin{document}
\maketitle

\begin{multicols}{2}

% Required points:
% - the considered version of the dataset (in terms of access date, as the dataset is updated very frequently), and the parts of the latter which have been considered,
% - how data have been organized,
% - the applied pre-processing techniques,
% - the considered algorithms and their implementations,
% - how the proposed solution scales up with data size,
% - a description of the experiments,
% - comments and discussion on the experimental results.

\textit{I declare that this material, which I now submit for assessment, is entirely my own work and has not been taken from the work of others, save and to the extent that such work has been cited and acknowledged within the text of my work. I understand that plagiarism, collusion, and copying are grave and serious offences in the university and accept the penalties that would be imposed should I engage in plagiarism, collusion or copying. This assignment, or any part of it, has not been previously submitted by me or any other person for assessment on this or any other course of study.}



\section{Introduction}
Recommender systems are an increasingly important feature in nowadays technological and economical development.
This project implements the most common base algorithm for recommending products, collaborative filtering, by applying it to a dataset from Yelp, a crowd-sourced platform for reviews about businesses in the USA and Canada areas.
The project is implemented as an interactive Python Jupyter notebook\footnote{\url{https://jupyter.org/}} and is built on the Spark engine\footnote{\url{https://spark.apache.org/}}.
It is meant to be run over an underlying distributed filesystem (e.g. HDFS) and it is potentially scalable.

The project is of course meant to be a \emph{proof of concept}, and would need many tweaks in order to be used in a production environment, however it shows (we hope) the fundamental logic behind the algorithm.

In this project report we go over the project setup (section \ref{setup}), the theory and the implementation of the algorithm (section \ref{collabfiltering}), the obtained results and their evaluation (section \ref{evaluation}), discuss the scalability aspects (section \ref{scalability}) and finally an overview of possible improvements and ideas (section \ref{futurework}).



\label{setup}
\section{Dataset and setup}



\label{collabfiltering}
\section{Collaborative filtering}



\label{evaluation}
\section{Results and evaluation}



\label{scalability}
\section{Scalability}



\label{futurework}
\section{Future work}


\end{multicols}

\end{document}
